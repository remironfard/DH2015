\newpage

\section*{Appendix: Web Components}
This appendix present technical details of our hypervideo implementation as web components.
We highlight that we not only optimized for reach on the consuming side---%
\emph{i.e.}, people watching the rehearsal hypervideos in their browsers---%
but also for the producing side---%
\emph{i.e.}, Web developers who collaborate with theaters and operas
to create such hypervideos in the first place and to put them online.
Web developers are used to mark up Webpages using HTML tags
like \texttt{<h1>}, \texttt{<p>}, \texttt{<strong>} \emph{etc.}
and to ``glue'' these tags functionally together using the JavaScript programming language
to finally control the look and feel of their Webpages with Cascading Stylesheets (CSS).
We were motivated to allow for the same tag-based mark-up mechanics
to also work for the creation of rich hypervideo applications.
Therefore, we have leveraged an emerging technology
that is currently being standardized by the World Wide Web Consortium (W3C)
called \emph{Web Components}.
Web Components is a~set of specifications, which let Web developers leverage
their HTML, CSS, and JavaScript knowledge to build widgets
that can be reused easily and reliably.\footnote{Web Components:
\url{http://www.chromium.org/blink/Web-components}}

%\todo{BE: si besoin de place, la partie qui suit peut être supprimée}
%%BE début partie pouvant être supprimée
%According to a~(recently discontinued) W3C Working Draft introductory document,%
%\footnote{Discontinued W3C Working Draft document:
%\url{http://www.w3.org/TR/2013/WD-components-intro-20130606/}~\cite{cooney2013Webcomponents}}
%the component model for the Web (``Web Components'') consists of five different pieces
%listed in the following.
%
%\begin{itemize}
%  \item \textbf{Imports} which defines how templates, decorators and custom elements are packaged and loaded as a~resource%
%  ~\cite{glazkov2014htmlimports}.
%  \item \textbf{Shadow DOM} which encapsulates a~DOM subtree for more reliable composition of user interface elements%
%  ~\cite{glazkov2014shadowdom}.    
%  \item \textbf{Custom Elements} which let authors define their own elements, with new tag names and new script interfaces%
%  ~\cite{glazkov2013customelements}.  
%  \item \textbf{Decorators} which apply templates based on CSS selectors to affect rich visual and behavioral changes to documents.
%  \item \textbf{Templates} which define chunks of inert markup that can be activated for use.  
%\end{itemize}
%%BE Fin partie

\noindent At the time of writing, partial native support for Web Components
has landed in a~number of Web browsers,
however, for the majority of browsers,
a~so-called \emph{polyfill} solution is still required.
A~polyfill  is a~piece of code that provides the technology
that developers expect the browser to provide natively in the near future.
We rely on the Polymer project\footnote{Polymer project:
\url{http://www.polymer-project.org/}}
that provides Web Components support for many browsers.
Polymer allows us to create reusable widgets that introduce a~number of new
custom HTML elements for our task of hypervideo creation.

We developed a~number of Web Components for the creation of hypervideos~\cite{steiner2014hypervideo}.
Web Components are behaviorally grouped together by a~common naming convention.
In Polymer, all element names have to start with the prefix \texttt{polymer-}
in order to create an own Web Components namespace
that distinguishes them from native HTML elements.

\paragraph*{\texttt{<polymer-hypervideo>}} is the parent element of all other elements.
    It accepts the attributes \texttt{src} for specifying a~set of
    space-separated video sources (to support different encodings),
    and---analog to the native HTML5 video attributes---%
    \texttt{width} and \texttt{height} for specifying the video's dimensions,
    then \texttt{poster} for specifying the video's poster frame, and finally \texttt{muted} to specify if the video should be initially muted.

\paragraph*{\texttt{<polymer-data-*>}} is a~set of data annotation elements
    that includes the two shorthand annotation types
    \texttt{<polymer-data-actor>} for annotating video actors and
    \texttt{<polymer-data-overlay>} for annotating visual overlays,
    and the generic \texttt{<polymer-data-annotation>} for other annotations.

\paragraph*{\texttt{<polymer-track-*>}} are the two textual elements
    \texttt{<polymer-track-chapters>} and \texttt{<polymer-track-subtitles>},
    which rely on WebVTT~\cite{pfeiffer2013webvtt} text tracks
    of type ``chapters'' and ``subtitles'' that they enrich~\cite{steiner2014webvtt} with
    automatically generated chapter thumbnails and a~full text view.
    
\paragraph*{\texttt{<polymer-visualization-*>}} currently provides the
    following two visualization elements
    \texttt{<polymer-visualization-timeline>} on the one hand and 
    \texttt{<polymer-visualization-toc>} on the other
    that create a~timeline view and a~table of contents
    that put all encountered \texttt{<polymer-track-*>}
    and \texttt{<polymer-data-*>} elements in a~temporal context.
